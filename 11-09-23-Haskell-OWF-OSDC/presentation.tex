\documentclass[10pt]{beamer}
\usepackage[frenchb]{babel}
\usepackage[utf8]{inputenc}
\usepackage{ucs}
\usetheme{Madrid}
\usepackage{amsmath,amssymb}
\newtheorem{thrm}{Théorème}

\title[Haskell]{Être paresseux avec classe\\
Une introduction à la programmation fonctionnelle pure en Haskell}
\author{Valentin Robert}
\institute{ENSEIRB / UCSD / INRIA}
\date{23 septembre 2011}

\begin{document}



\begin{frame}
\titlepage
\end{frame}



\begin{frame}[fragile]
\frametitle{Programmation fonctionnelle : pourquoi ?}

\begin{block}{Motivations}
\begin{itemize}
\item Systèmes à très grande échelle
\item Concurrence
\item Parallélisme
\item Productivité
\end{itemize}
\end{block}

\pause

\begin{block}{Qui ? Où ?}
\begin{itemize}
\item Microsoft (GHC)
\item Java (Scala, Clojure)
\item Tout le monde ?
\end{itemize}
\end{block}

\end{frame}



\begin{frame}
\frametitle{Qu'est-ce qu'Haskell ?}

\begin{itemize}

\item Haskell est fonctionnel
\pause

\item Haskell est paresseux
\pause

\item Haskell est pur
\pause

\item Haskell est d'ordre supérieur
\pause

\item Haskell est typé statiquement (et fortement)

\end{itemize}
\end{frame}



\begin{frame}
\frametitle{Que peut-on faire en Haskell ?}

\pause

Tout !

\pause

\begin{itemize}

\item Frameworks web (Happstack, Yesod)
\pause

\item Implémenter des langages (GHC, Pugs)
\pause

\item Gestionnaire de fenêtre xmonad
\pause

\item Jeu vidéo (Nikki and the Stars)
\pause

\item ... de nombreuses bibliothèques sont disponibles ! (Hackage)
\pause

\item ... ainsi qu'une interface de fonctions étrangères !

\end{itemize}

\end{frame}



\begin{frame}[fragile]
\frametitle{Programme Haskell 1}

\begin{verbatim}
nats = [0, 1..]

first10nats = take 10 nats

main = do print "Hello World!"
          print first10nats
\end{verbatim}

\end{frame}



\begin{frame}[fragile]
\frametitle{Évaluation paresseuse, types}

\begin{verbatim}
-- take :: Integer -> [a] -> [a]

first10 :: [a] -> [a]
first10 = take 10

first5  list = take 5 (take 10 list)

first5' list = take 10 (take 5 list)

main :: IO ()
main = do print (first5 [0, 1..])
\end{verbatim}

\end{frame}



\begin{frame}[fragile]
\frametitle{Quicksort}

\begin{verbatim}
-- Les listes en Haskell :
-- []               représente la liste vide ("nil")
-- 1:[]             représente la liste à un élément [1]
-- 3:(2:(1:[]))     représente la liste [3, 2, 1]
-- [2, 1]           également (surce syntaxique pour 2:1:[])

quicksort :: Ord a => [a] -> [a]
quicksort []     = []
quicksort (p:xs) = (quicksort lesser) ++ [p] ++ (quicksort greater)
    where
        lesser  = filter (< p) xs
        greater = filter (>= p) xs
\end{verbatim}

\pause

\begin{verbatim}
main = do dirsElts <- getDirectoryContents "."
          print (quicksort dirElts)
          print (quicksort dirElts) -- pas besoin de recalculer !
\end{verbatim}

\end{frame}



\begin{frame}[fragile]
\frametitle{Entrées-sorties paresseuses}
\begin{verbatim}
import qualified Data.ByteString.Lazy as B
import qualified Data.ByteString.Char8 as C

main = do byteString <- B.readFile "/dev/urandom"
          let infiniteString = C.unpack byteString
          let tenChars = take 10 (filter isAlphaNum string)
          putStrLn tenChars
\end{verbatim}

\begin{verbatim}
ghci> main
ZwBÛAãÅWóÕ
ghci> main
Fh0iXXïZìæ
\end{verbatim}
\end{frame}



\begin{frame}[fragile]
\frametitle{Les fonctions pures sont simples à tester !}
\begin{block}{Code}
\begin{verbatim}
focusPrev :: CircList a → CircList a

focusNext :: CircList a → CircList a
\end{verbatim}
\end{block}
\pause
\begin{block}{Propriété}
\begin{verbatim}
prop_focus :: CircList a → Bool
prop_focus s = (s == focusNext (focusPrev s))
\end{verbatim}
\end{block}
\pause
\begin{block}{Quickcheck}
\begin{verbatim}
> quickCheck prop_focus
OK, passed 100 tests.
\end{verbatim}
\end{block}
\end{frame}



\begin{frame}[fragile]
\frametitle{Scripter en Haskell}
\begin{block}{Partie impure}
\begin{verbatim}
main :: IO ()
main = do as <- getArgs
          mapM_ process as

process :: String → IO ()
process file =
  do cts <- readFile file
     let tests = getTests cts
     if null tests then
       putStrLn (file ++ ": no properties to check")
     else do
       writeFile "script" $
         unlines ([":l " ++ file] ++ concatMap makeTest tests)
       system ("ghci -v0 < script")
       return ()
\end{verbatim}
\end{block}
\end{frame}



\begin{frame}[fragile]
\frametitle{Scripter en Haskell}
\begin{block}{Partie pure}
\begin{verbatim}
getTests :: String → [String]
getTests cts = nub
               . filter ("prop_" `isPrefixOf`)
               . map (fst . head . lex)
               $ lines cts

makeTest :: String → [String]
makeTest test = ["putStr \"" ++ test ++ ": \"", “quickCheck " ++
                  test]
\end{verbatim}
\end{block}
\end{frame}



\begin{frame}
\centerline{Merci de votre présence et attention !}

\begin{figure}
\includegraphics[scale=0.4]{haskell_fr_logo.png}
\end{figure}

\centerline{Nouveau ! \url{http://haskell.fr}}

\centerline{Nouveau ! \url{http://lyah.haskell.fr}}
\end{frame}



\end{document}
